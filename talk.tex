\documentclass[dvipsnames]{beamer}

\usepackage[utf8]{inputenc}
\usepackage{url}
\usepackage{amssymb}
\usepackage{amsmath}
\usepackage{nccmath}
\usepackage{amsbsy}
\usepackage{mathbbol}
\usepackage{mathtools}

% For a wider page
\usepackage{changepage}

% Diagrams
\usepackage{tikz}

% Mathbb doesn't support digits
\usepackage{bbm}

% Example code
\usepackage{listings}

% Inference rules
\usepackage{mathpartir}

% Multiple columns
\usepackage{multicol}

% Newline inside table
\usepackage{makecell}

% Square bullet points
\newcommand{\sitem}{\item[\raisebox{.45ex}{\rule{.6ex}{.6ex}}]}

% Abbreviations
\newcommand{\lambdacalc}{$\lambda$-calculus}
\newcommand{\picalc}{$\pi$-calculus}
\newcommand{\Picalc}{$\pi$-Calculus}

% Typing rules
\newcommand{\stacked}[1]{\mprset{flushleft} \inferrule*{}{#1}}
\newcommand{\datatype}[2]{{\mprset{fraction={===}} \inferrule{#1}{#2}}}
\newcommand{\rulename}[1]{{\tiny \textsc{#1}}}

% Constructors
\newcommand{\PO}{\textbf{0}}
\newcommand{\comp}[2]{#1 \parallel #2}
\newcommand{\new}[2]{(\boldsymbol{\nu} \, #1 \, #2) \,}
\newcommand{\send}[2]{#1 !\langle #2 \rangle .}
\newcommand{\recv}[2]{#1 ?( #2 ) .}
\newcommand{\branch}[3]{#1 \triangleright \{ #2 \}_{#3}}
\newcommand{\select}[2]{#1 \triangleleft #2 .}
\newcommand{\type}[1]{\mathtt{#1}}
\newcommand{\tend}[0]{\type{end}}
\newcommand{\trecv}[1]{\type{?} #1 \type{.}}
\newcommand{\tsend}[1]{\type{!} #1 \type{.}}
\newcommand{\tbranch}[2]{\type{\&} \{ #1 \}_{#2}}
\newcommand{\tselect}[2]{\type{\oplus} \{ #1 \}_{#2}}
\newcommand{\tchan}[1]{\type{C[} #1 \type{]}}
\newcommand{\dual}[1]{\overline{#1}}
\newcommand{\is}[0]{\triangleq}
\newcommand{\types}[0]{\vdash}

\newcommand{\reducesto}{\Longrightarrow}
\newcommand{\subst}[3]{{#1 \left[ #2 \rightarrow #3 \right]}}


\usetheme[sectionpage=none,numbering=none]{metropolis}
\setbeamertemplate{title separator}{}
\date{}


\title{session types}

\begin{document}
  \maketitle

  \begin{frame}{overview}
    \begin{center}
      data types to structure data;\\
      session types to structure communication
    \end{center}
  \end{frame}

  \begin{frame}{key ingredients}
    \begin{description}
      \setlength\itemsep{1em}
    \item[endpoints] each channel has two endpoints
    \item[linearity] channel endpoints are uniquely owned
    \item[duality] relation between types of opposite endpoints
    \item[structure] session types follow the structure of communication
    \item[change] session types evolve as communication occurs
    \end{description}
  \end{frame}

  \begin{frame}{properties}
    \begin{block}{communication safety}
      the exchanged data is of the expected type
    \end{block}
    \begin{block}{session fidelity}
      communication is structured as per the session type
    \end{block}
    \begin{block}{privacy}
      only the two communicating parties can listen on a channel
    \end{block}
    \begin{block}{no race conditions}
      no two processes race to obtain the same resource
    \end{block}
  \end{frame}

  \begin{frame}{types}
    \textbf{coinductive}
    \setlength{\abovedisplayskip}{0pt}
    \setlength{\belowdisplayskip}{0pt}
    \setlength{\abovedisplayshortskip}{0pt}
    \setlength{\belowdisplayshortskip}{0pt}
    \begin{align*}
      S ::=& \; \tend      &&\text{(termination)}    \\ 
      |& \; \trecv{T}S     &&\text{(input)}       \\ 
      |& \; \tsend{T}S     &&\text{(output)}       \\           
      |& \; \tbranch{l_i : S_i}{i \in I} &&\text{(branch)}       \\ 
      |& \; \tselect{l_i : S_i}{i \in I} &&\text{(select)}       \\           
      \\
      T ::=& \; \tchan{S} &&\text{(session type)} \\
      | & \; \ldots       &&\text{(base data types)} \\
    \end{align*}
    \begin{align*}
      Srv & \is \trecv{\mathtt{\mathbb{N}}} \tselect{\mathtt{win} : \tend ; \; \mathtt{miss} : Srv}{} \\
      Cln & \is \tsend{\mathtt{\mathbb{N}}} \tbranch{\mathtt{win} : \tend ; \; \mathtt{miss} : Cln}{} \\
    \end{align*}
  \end{frame}

  \begin{frame}{duality}
    \begin{equation*}
      \begin{aligned}
        \dual{\tend}          &\is \tend\\
        \dual{\trecv{T}S}     &\is \tsend{T}\dual{S}\\
        \dual{\tsend{T}S}     &\is \trecv{T}\dual{S}\\
        \dual{\tbranch{l_i : S_i}{i \in I}} &\is \tselect{l_i : \dual{S_i}}{i \in I}\\
        \dual{\tselect{l_i : S_i}{i \in I}} &\is \tbranch{l_i : \dual{S_i}}{i \in I}\\
      \end{aligned}
    \end{equation*}
  \end{frame}

  \begin{frame}{languages for concurrent communication}
    \begin{itemize}
      \setlength\itemsep{1em}
      \sitem \picalc{}
      \sitem \textbf{G}ay \textbf{V}asconcelos
      \sitem Typestate
      \sitem \textbf{C}alculus of \textbf{C}ommunicating \textbf{S}ystems (Milner)
      \sitem \textbf{C}ommmunicating \textbf{S}equential \textbf{P}rocesses (Hoare)
    \end{itemize}
  \end{frame}
  
  \begin{frame}{example: the \picalc{}}
    \begin{itemize}
      \setlength\itemsep{1em}
      \sitem a process algebra
      \sitem processes communicate over channels
      \sitem channels are sent over channels (\emph{channel mobility})
      \sitem communication drives computation
    \end{itemize}
  \end{frame}

  \begin{frame}{syntax}
    \textbf{coinductive}
    \begin{align*}
        P, Q ::=& \; \PO                     &&\text{(termination)}    \\
        |& \; \new{x}{y}P                    &&\text{(scope restriction)} \\
        |& \; \comp{P}{Q}                    &&\text{(parallel composition)} \\
        |& \; \recv{x}{y}P                   &&\text{(input)}       \\
        |& \; \send{x}{v}P                   &&\text{(output)}       \\
        |& \; \branch{x}{l_i : P_i}{i \in I} &&\text{(branch)}       \\
        |& \; \select{x}{l_j}P               &&\text{(select)}       \\
        |& \; \ldots                         &&\text{(base constructs)} \\
        \\
        v ::=& \; x, y && \text{(channels)}\\
        | & \; \ldots && \text{(base values)}\\
    \end{align*}
  \end{frame}

  \begin{frame}{operational semantics}
    \begin{mathpar}
      \inferrule
      { }
      {\new{x}{y} (\comp{\send{x}{v}P}{\recv{y}{u}Q}) \\ \reducesto \\ \new{x}{y} (\comp{P}{\subst{Q}{u}{v}})}
      {\rulename{Comm}}\\

      \inferrule
      {i \in J}
      {\new{x}{y} (\comp{\select{x}{l_i}P}{\branch{y}{l_j : Q_j}{j \in J}}) \\ \reducesto \\ \new{x}{y} (\comp{P}{Q_i})}
      {\rulename{Case}}\\

      \inferrule
      {P \reducesto P'}
      {\comp{P}{Q} \reducesto \comp{P'}{Q}}
      {\rulename{Par}}

      \inferrule
      {P \reducesto P'}
      {\new{x}{y} P \reducesto \new{x}{y} P'}
      {\rulename{Res}}

      \inferrule
      {P' \equiv{} P \\ P \reducesto Q \\ Q \equiv{} Q'}
      {P' \reducesto Q'}
      {\rulename{StructCong}}
    \end{mathpar}
  \end{frame}

  \begin{frame}{structural congruence}
    \begin{center}
      congruent equivalence closure of
    \end{center}

    \begin{mathpar}
      \inferrule
      { }
      {\comp{P}{(\comp{Q}{R})} \equiv \comp{(\comp{P}{Q})}{R}}
      {\rulename{CompAssoc}}

      \inferrule
      { }
      {\comp{P}{Q} \equiv \comp{Q}{P}}
      {\rulename{CompComm}}
      
      \inferrule
      { }
      {\comp{P}{\PO} \equiv P}
      {\rulename{CompEnd}}
      
      \inferrule
      { }
      {\new{x}{y} \PO \equiv \PO}
      {\rulename{ScopeEnd}}
      
      \inferrule
      {\{x , y \} \cap FV(Q) = \emptyset}
      {\new{x}{y} (\comp{P}{Q}) \equiv \comp{\new{x}{y} P}{Q}}
      {\rulename{ScopeExt}}

      \inferrule
      { }
      {\new{x}{y} \new{x'}{y'} P \equiv \new{x'}{y'} \new{x}{y} P}
      {\rulename{ScopeComm}}
    \end{mathpar}
  \end{frame}
  \note[itemize]{
    \sitem Parallel composition is associative, commutative and has $\PO$ as neutral element
    \sitem Garbage collection
    \sitem Scope extrusion (keep a proof $uQ$ alongside)
    \sitem Commutativity of scopes
  }

  \begin{frame}{typing rules}
    \begin{mathpar}
      \inferrule
      {un(\Gamma)}
      {\Gamma \types \PO}
      {\rulename{End}}

      \inferrule
      {\Gamma = \Gamma_P + \Gamma_Q \\ \Gamma_P \types P \\ \Gamma_Q \types Q}
      {\Gamma \types \comp{P}{Q}}
      {\rulename{Par}}

      \inferrule
      {\Gamma , x : S , y : \dual{S} \types P}
      {\Gamma \types \new{x}{y} P}
      {\rulename{Res}}

      \inferrule
      {\Gamma , x : S , v : T \types P}
      {\Gamma , x : \trecv{T}{S} \types \recv{x}{v} P}
      {\rulename{In}}

      \inferrule
      {\Gamma , x : S \types P}
      {\Gamma , x : \tsend{T}S , v : T \types \send{x}{v} P}
      {\rulename{Out}}
    \end{mathpar}
  \end{frame}

  \begin{frame}{notions of progress}
    \begin{block}{\textcolor{red}{termination}}
      every process reduces to $\PO$
    \end{block}

    \begin{block}{\textcolor{red}{lock freedom}}
      guaranteed communication
    \end{block}

    \begin{block}{\textcolor{red}{deadlock freedom}}
      do not get stuck (on cyclic channel dependencies)
    \end{block}
  \end{frame}

  \begin{frame}{extensions}
    \begin{block}{polymorphism}\hfill\\
      $\forall T \Rightarrow \trecv{T} \tsend{T} \tend$
    \end{block}
    \begin{block}{subtyping}\hfill\\
      $
        \inferrule
        {I \supseteq J \\ S_i <: S'_j}
        {\tselect{l_i : S_i}{i \in I} <: \tselect{l_j : S'_j}{j \in J}}
        {}
      $
    \end{block}
  \end{frame}

  \begin{frame}{extensions}
    \begin{block}{higher order}
      \begin{fleqn}[0pt]
        \setlength{\abovedisplayskip}{0pt}
        \setlength{\belowdisplayskip}{0pt}
        \begin{alignat*}{3}
          & && \new{x}{y} (\comp {\send{x}{ \mathtt{quote} \, \mathbf{P}} Q} {\recv{y}{\mathbf{P'}}{ \mathtt{unquote} \, \mathbf{P'}}}) & \\
          & \reducesto && \new{x}{y} (\comp{Q}{\mathtt{unquote} \, \mathtt{quote} \, P}) & \\
          & \reducesto && \new{x}{y} (\comp{Q}{P}) &
        \end{alignat*}
      \end{fleqn}
    \end{block}
    \begin{block}{dependent session types}\hfill\\
      \begin{itemize}
        \sitem $\trecv{\mathbb{N}} \lambda n \rightarrow S$
        \sitem \framebox{$\trecv{\tchan{\trecv{\mathbb{N}} \tsend{\mathbb{N}} \tend}} \lambda c \rightarrow S$}?
      \end{itemize}
    \end{block}
  \end{frame}

  \begin{frame}{embeddings and correspondences}
    \begin{block}{to the linear \picalc{}}\hfill\\
      $\Gamma \vdash \mathbf{P} \implies [\![ \Gamma ]\!] \vdash [\![ \mathbf{P} ]\!]$
    \end{block}

    \begin{block}{from effect systems}\hfill\\
      $\Gamma \vdash e : T , F \implies [\![ \Gamma ]\!]; \, x : \tsend {[\![ T ]\!]} \tend , \, y : [\![ F ]\!] \vdash [\![ e ]\!]$
    \end{block}

    \begin{block}{correspondence with linear logic}
    \end{block}
  \end{frame}

  \begin{frame}
    \centering
    \Huge{thank you!}\\
    \Huge{questions?}\\
  \end{frame}
\end{document}