\documentclass[dvipsnames]{beamer}

\usepackage[utf8]{inputenc}
\usepackage{url}
\usepackage{amssymb}
\usepackage{amsmath}
\usepackage{amsbsy}
\usepackage{mathbbol}
\usepackage{mathtools}

% For a wider page
\usepackage{changepage}

% Diagrams
\usepackage{tikz}

% Mathbb doesn't support digits
\usepackage{bbm}

% Example code
\usepackage{listings}

% Inference rules
\usepackage{mathpartir}

% Multiple columns
\usepackage{multicol}

% Newline inside table
\usepackage{makecell}

% Square bullet points
\newcommand{\sitem}{\item[\raisebox{.45ex}{\rule{.6ex}{.6ex}}]}

% Abbreviations
\newcommand{\lambdacalc}{$\lambda$-calculus}
\newcommand{\picalc}{$\pi$-calculus}
\newcommand{\Picalc}{$\pi$-Calculus}

% Typing rules
\newcommand{\stacked}[1]{\mprset{flushleft} \inferrule*{}{#1}}
\newcommand{\datatype}[2]{{\mprset{fraction={===}} \inferrule{#1}{#2}}}
\newcommand{\rulename}[1]{{\tiny \textsc{(#1)}}}

% Constructors
\newcommand{\PO}{\textbf{0}}
\newcommand{\comp}[2]{#1 \parallel #2}
\newcommand{\new}[2]{(\boldsymbol{\nu} \, #1 \, #2) \,}
\newcommand{\send}[2]{#1 !\langle #2 \rangle .}
\newcommand{\recv}[2]{#1 ?( #2 ) .}
\newcommand{\branch}[3]{#1 \triangleright \{ #2 \}_{#3}}
\newcommand{\select}[2]{#1 \triangleleft #2 .}
\newcommand{\type}[1]{\mathtt{#1}}
\newcommand{\tend}[0]{\type{end}}
\newcommand{\trecv}[1]{\type{?} #1 \type{.}}
\newcommand{\tsend}[1]{\type{!} #1 \type{.}}
\newcommand{\tbranch}[2]{\type{\&} \{ #1 \}_{#2}}
\newcommand{\tselect}[2]{\type{\oplus} \{ #1 \}_{#2}}
\newcommand{\dual}[1]{\overline{#1}}
\newcommand{\is}[0]{\triangleq}
\newcommand{\types}[0]{\vdash}
\newcommand{\csplit}[0]{\otimes}

\newcommand{\reducesto}{\Longrightarrow}
\newcommand{\subst}[3]{{#1 \left[ #2 \rightarrow #3 \right]}}

\newcommand{\slidetitle}[2]{#2 \hspace*{\fill} [#1]}

\usetheme[sectionpage=none,numbering=none]{metropolis}
\setbeamertemplate{title separator}{}
\date{}


\title{Introduction to the\\ Session Typed \Picalc{}}
\author{\textbf{Uma Zalakain}\\ University of Glasgow}

\begin{document}
  \maketitle

  \begin{frame}
    \begin{itemize}
      \setlength\itemsep{1em}
      \sitem untyped \picalc{}
        \begin{itemize}
          \sitem motivation
          \sitem syntax
          \sitem operational semantics
        \end{itemize}
      \sitem session typed \picalc{}
        \begin{itemize}
          \sitem motivation
          \sitem types
          \sitem type system
          \sitem properties
        \end{itemize}
    \end{itemize}
  \end{frame}

  \begin{frame}{\slidetitle{Untyped \Picalc{}}{Motivation}}
    \begin{itemize}
      \setlength\itemsep{1em}
      \sitem a process algebra
      \sitem models concurrent and communicating processes
      \sitem processes communicate over channels
      \sitem channels are sent over channels (\emph{channel mobility})
      \sitem communication drives computation
    \end{itemize}
  \end{frame}
  
  \begin{frame}{\slidetitle{Untyped \Picalc{}}{Syntax}}
    \begin{align*}
        P, Q ::=& \; \PO                     &&\text{(termination)}    \\ 
        |& \; \new{x}{y}P                    &&\text{(scope restriction)} \\ 
        |& \; \comp{P}{Q}                    &&\text{(parallel composition)} \\ 
        |& \; \recv{x}{v}P                   &&\text{(input)}       \\ 
        |& \; \send{x}{v}P                   &&\text{(output)}       \\           
        |& \; \branch{x}{l_i : P_i}{i \in I} &&\text{(branch)}       \\ 
        |& \; \select{x}{l_j}P               &&\text{(select)}       \\           
        |& \; \ldots                         &&\text{(base constructs)} \\
        \\
        v ::=& \; x && \text{(channels)}\\
        | & \; \ldots && \text{(base values)}\\
    \end{align*}
  \end{frame}

  \begin{frame}{\slidetitle{Untyped \Picalc{}}{Example}}
    \begin{alignat*}{3}
        & Srv \, x \, \is \recv{x}{n} && \mathtt{if} \, && n \, \mathtt{\stackrel{?}{=}} \, \mathtt{13} \\
        & && \mathtt{then} \; && \select{x}{\mathtt{win}} \PO \\
        & && \mathtt{else} \; && \select{x}{\mathtt{miss}} S \, x
    \end{alignat*}
    \begin{alignat*}{3}
        & Cln \, x \, n \is \send{x}{n} \branch{x}{ && \mathtt{win} : \PO; \\
        & && \mathtt{miss} : C \, x \, (1+n)}{}
    \end{alignat*}
  \end{frame}

  \begin{frame}{\slidetitle{Untyped \Picalc{}}{Reduction Relation}}
    \begin{mathpar}
      \inferrule
      { }
      {\new{x}{y} \comp{\send{x}{v}P}{\recv{y}{u}Q} \\ \reducesto \\ \new{x}{y} \comp{P}{\subst{Q}{u}{v}}}
      {\rulename{Comm}}\\

      \inferrule
      {i \in J}
      {\new{x}{y} \comp{\select{x}{l_i}P}{\branch{y}{l_j : Q_j}{j \in J}} \\ \reducesto \\ \new{x}{y} \comp{P}{Q_i}}
      {\rulename{Case}}\\

      \inferrule
      {P \reducesto P'}
      {\comp{P}{Q} \reducesto \comp{P'}{Q}}
      {\rulename{Par}}

      \inferrule
      {P \reducesto P'}
      {\new{x}{y} P \reducesto \new{x}{y} P'}
      {\rulename{Res}}

      \inferrule
      {P \equiv{} P' \\ P' \reducesto P''}
      {P \reducesto P''}
      {\rulename{StructCong}}
    \end{mathpar}
  \end{frame}

  \note[itemize]{
    \sitem Same scope before and after reduction
    \sitem Common channel $i$
    \sitem Goes under parallel composition and scope restriction
  }


  \begin{frame}{\slidetitle{Untyped \Picalc{}}{Structural Congruence}}
    \begin{center}
      congruent equivalence closure of
    \end{center}

    \begin{mathpar}
      \inferrule
      { }
      {\comp{P}{(\comp{Q}{R})} \equiv \comp{(\comp{P}{Q})}{R}}
      {\rulename{CompAssoc}}

      \inferrule
      { }
      {\comp{P}{Q} \equiv \comp{Q}{P}}
      {\rulename{CompComm}}
      
      \inferrule
      { }
      {\comp{P}{\PO} \equiv P}
      {\rulename{CompEnd}}
      
      \inferrule
      { }
      {\new{x}{y} \PO \equiv \PO}
      {\rulename{ScopeEnd}}
      
      \inferrule
      {x \not\in FV(Q) \\ y \not\in FV(Q)}
      {\new{x}{y} (\comp{P}{Q}) \equiv \comp{\new{x}{y} P}{Q}}
      {\rulename{ScopeExt}}

      \inferrule
      { }
      {\new{x}{y} \new{x'}{y'} P \equiv \new{x'}{y'} \new{x}{y} P}
      {\rulename{ScopeComm}}
    \end{mathpar}
  \end{frame}
  \note[itemize]{
    \sitem Parallel composition is associative, commutative and has $\PO$ as neutral element
    \sitem Garbage collection
    \sitem Scope extrusion (keep a proof $uQ$ alongside)
    \sitem Commutativity of scopes
  }

  \begin{frame}{\slidetitle{Untyped \Picalc{}}{Example}}
    \vspace{-1em}
    \begin{alignat*}{2}
      & \new{x}{y} && \comp{Srv \, x}{Cln \, y \, 0} \\
      & && \Downarrow \\
      & \new{x}{y} && \comp{\recv{x}{n} \mathtt{if} \; n \, \mathtt{\stackrel{?}{=}} \, \mathtt{13} \; \mathtt{then} \; \select{x}{\mathtt{win}} \PO \; \mathtt{else} \; \select{x}{\mathtt{miss}} Srv \, x}{\\
      & && \send{y}{0} \branch{y}{ \mathtt{win} : \PO; \mathtt{miss} : Cln \, y \, 1}{}} \\
      & && \Downarrow \\
      & \new{x}{y} && \comp{\mathtt{if} \; 0 \, \mathtt{\stackrel{?}{=}} \, \mathtt{13} \; \mathtt{then} \; \select{x}{\mathtt{win}} \PO \; \mathtt{else} \; \select{x}{\mathtt{miss}} Srv \, x}{\\
      & && \branch{y}{ \mathtt{win} : \PO; \mathtt{miss} : Cln \, y \, 1}{}} \\
      & && \Downarrow \\
      & \new{x}{y} && \comp{\select{x}{\mathtt{miss}} Srv \, x}{\branch{x}{ \mathtt{win} : \PO; \mathtt{miss} : Cln \, x \, 1}{}} \\
      & && \Downarrow \\
      & \new{x}{y} && \comp{Srv \, x}{Cln \, y \, 1} \\
    \end{alignat*}{2}
  \end{frame}

  \begin{frame}{\slidetitle{Untyped \Picalc{}}{Example}}
    \vspace{-1em}
    \begin{alignat*}{2}
      & \new{x}{y} && \comp{Srv \, x}{Cln \, y \, 13} \\
      & && \Downarrow \\
      & \new{x}{y} && \comp{\recv{x}{n} \mathtt{if} \; n \, \mathtt{\stackrel{?}{=}} \, \mathtt{13} \; \mathtt{then} \; \select{x}{\mathtt{win}} \PO \; \mathtt{else} \; \select{x}{\mathtt{miss}} Srv \, x}{\\
      & && \send{y}{13} \branch{y}{ \mathtt{win} : \PO; \mathtt{miss} : Cln \, y \, 14}{}} \\
      & && \Downarrow \\
      & \new{x}{y} && \comp{\mathtt{if} \; 13 \, \mathtt{\stackrel{?}{=}} \, \mathtt{13} \; \mathtt{then} \; \select{x}{\mathtt{win}} \PO \; \mathtt{else} \; \select{x}{\mathtt{miss}} Srv \, x}{\\
      & && \branch{y}{ \mathtt{win} : \PO; \mathtt{miss} : Cln \, y \, 14}{}} \\
      & && \Downarrow \\
      & \new{x}{y} && \comp{\select{x}{\mathtt{win}} \PO}{\branch{y}{ \mathtt{win} : \PO; \mathtt{miss} : Cln \, y \, 14}{}} \\
      & && \Downarrow \\
      & \new{x}{y} && \comp{\PO}{\PO} \\
    \end{alignat*}
  \end{frame}
  

  \begin{frame}{\slidetitle{Session Typed \Picalc{}}{Overview}}
    \begin{center}
      data types to structure data;\\
      session types to structure communication
    \end{center}

    \begin{block}{communication safety}
      the exchanged data is of the expected type
    \end{block}
    \begin{block}{session fidelity}
      communication is structured as per the session type
    \end{block}
    \begin{block}{privacy}
      only the two communicating parties can listen on a channel
    \end{block}
    \begin{block}{no race conditions}
      no two processes race to obtain the same resource
    \end{block}
  \end{frame}

  \begin{frame}{\slidetitle{Session Typed \Picalc{}}{Characteristics}}
    \begin{description}
      \setlength\itemsep{1em}
      \item[endpoints] each channel has two endpoints
      \item[linearity] channel endpoints are uniquely owned
      \item[duality] relation between types of opposite endpoints
      \item[structure] session types follow the structure of communication
      \item[change] session types evolve as communication occurs
    \end{description}
  \end{frame}

  \begin{frame}{\slidetitle{Session Typed \Picalc{}}{Types}}
    \setlength{\abovedisplayskip}{0pt}
    \setlength{\belowdisplayskip}{0pt}
    \setlength{\abovedisplayshortskip}{0pt}
    \setlength{\belowdisplayshortskip}{0pt}
    \begin{align*}
      Srv \is \trecv{\mathtt{Int}} \tselect{\mathtt{win} : \tend ; \; \mathtt{miss} : S}{} \\
      Cln \is \tsend{\mathtt{Int}} \tbranch{\mathtt{win} : \tend ; \; \mathtt{miss} : C}{} \\
    \end{align*}
    \begin{align*}
      S ::=& \; \tend      &&\text{(termination)}    \\ 
      |& \; \trecv{T}S     &&\text{(input)}       \\ 
      |& \; \tsend{T}S     &&\text{(output)}       \\           
      |& \; \tbranch{l_i : S_i}{i \in I} &&\text{(branch)}       \\ 
      |& \; \tselect{l_i : S_i}{i \in I} &&\text{(select)}       \\           
      \\
      T ::=& \; S      &&\text{(session type)} \\
      | & \; \ldots      &&\text{(base data types)} \\
    \end{align*}
  \end{frame}

  \begin{frame}{\slidetitle{Session Typed \Picalc{}}{Duality}}
    \begin{equation*}
      \begin{aligned}
        &\dual{\tend}          &&\is \tend\\
        &\dual{\trecv{T}S}     &&\is \tsend{T}\dual{S}\\
        &\dual{\tsend{T}S}     &&\is \trecv{T}\dual{S}\\
        &\dual{\tbranch{l_i : S_i}{i \in I}} &&\is \tselect{l_i : \dual{S_i}}{i \in I}\\
        &\dual{\tselect{l_i : S_i}{i \in I}} &&\is \tbranch{l_i : \dual{S_i}}{i \in I}\\
      \end{aligned}
    \end{equation*}
  \end{frame}

  \begin{frame}{\slidetitle{Session Typed \Picalc{}}{Typing Judgments}}
    \begin{mathpar}
      \datatype
      { }
      {\Gamma \ni e : S}
      {\rulename{Contains}}

      \datatype
      { }
      {\Gamma \types P}
      {\rulename{Types}}
    \end{mathpar}
  \end{frame}

  \begin{frame}{\slidetitle{Session Typed \Picalc{}}{Typing Rules}}
    \begin{mathpar}
      \small
      
      \inferrule
      {un(\Gamma)}
      {\Gamma , x : S \ni x : S}
      {\rulename{Var}}

      \inferrule
      {\Gamma \ni x : \tend}
      {\Gamma \types \PO}
      {\rulename{End}}

      \inferrule
      {\Gamma = \Gamma_P \csplit \Gamma_Q \\ \Gamma_P \types P \\ \Gamma_Q \types Q}
      {\Gamma \types \comp{P}{Q}}
      {\rulename{Par}}

      \inferrule
      {\Gamma , x : S , y : \dual{S} \types P}
      {\Gamma \types \new{x}{y} P}
      {\rulename{Res}}

      \inferrule
      {\Gamma = \Gamma_x \csplit \Gamma_P \\ \Gamma_x \ni x : \trecv{T}S \\ \Gamma_P , x : S , v : T \types P}
      {\Gamma \types \recv{x}{v} P}
      {\rulename{In}}

      \inferrule
      {\Gamma = \Gamma_x \csplit \Gamma_v \csplit \Gamma_P \\
        \stacked{
          \Gamma_x \ni x : \trecv{T}S \\\\
          \Gamma_v \ni v : S} \\
        \Gamma_P \types P}
      {\Gamma \types \send{x}{v} P}
      {\rulename{Out}}
    \end{mathpar}
  \end{frame}

  \begin{frame}{\slidetitle{Session Typed \Picalc{}}{Encodings}}
    correspondence with linear picalc
  \end{frame}

  \begin{frame}{\slidetitle{Session Typed \Picalc{}}{Extensions}}
    \begin{block}{higher order}
      channels transmit processes
    \end{block}
    \begin{block}{polymorphism}
      abstractions in session types
    \end{block}
    \begin{block}{dependent channel types}
      session types depend on received values
    \end{block}
  \end{frame}

  \begin{frame}{\slidetitle{Session Typed \Picalc{}}{Further Properties}}
    \begin{block}{deadlock freedom}

    \end{block}
    \begin{block}{lock freedom}
    \end{block}
    \begin{block}{progress}
    \end{block}
  \end{frame}

  \begin{frame}{Acknowledgements}
    Ornela Dardha
  \end{frame}
\end{document}