\documentclass[dvipsnames]{beamer}

\usepackage[utf8]{inputenc}
\usepackage{url}
\usepackage{amssymb}
\usepackage{amsmath}
\usepackage{amsbsy}
\usepackage{mathbbol}
\usepackage{mathtools}

\usepackage{minibox}

% For a wider page
\usepackage{changepage}

% Diagrams
\usepackage{tikz}

% Mathbb doesn't support digits
\usepackage{bbm}

% Example code
\usepackage{listings}

% Inference rules
\usepackage{mathpartir}

% Multiple columns
\usepackage{multicol}

% Newline inside table
\usepackage{makecell}

% Square bullet points
\newcommand{\sitem}{\item[\raisebox{.45ex}{\rule{.6ex}{.6ex}}]}

% Abbreviations
\newcommand{\lambdacalc}{$\lambda$-calculus}
\newcommand{\picalc}{$\pi$-calculus}
\newcommand{\Picalc}{$\pi$-Calculus}

% Typing rules
\newcommand{\stacked}[1]{\mprset{flushleft} \inferrule*{}{#1}}
\newcommand{\datatype}[2]{{\mprset{fraction={===}} \inferrule{#1}{#2}}}

\newcommand{\type}[1]{\textcolor{BlueViolet}{\operatorname{#1}}}
\newcommand{\constr}[1]{\textcolor{BurntOrange}{\operatorname{#1}}}
\newcommand{\func}[1]{\textcolor{OliveGreen}{\operatorname{#1}}}

% Constructors
\newcommand{\PO}{\constr{\mathbb{0}}}
\newcommand{\comp}[2]{#1 \; \constr{\parallel} \; #2}
\newcommand{\new}{\constr{\boldsymbol{\nu}} \;}
\newcommand{\send}[2]{#1 \; \constr{\langle} \; #2 \;\constr{\rangle} \;}
\newcommand{\sendp}[2]{#1 \; \constr{\langle} \; #2 \; \constr{\rangle} \; . \;}
\newcommand{\recv}[1]{#1 \; \constr{\mathbb{()}} \;}
\newcommand{\recvp}[2]{#1 \; \constr{(} \; #2 \; \constr{)} \; . \; }
\newcommand{\suc}{\constr{\scriptstyle 1+}}
\newcommand{\nzero}{\constr{\scriptstyle 0}}
\newcommand{\unit}{\constr{\mathbbm{1}}}
\newcommand{\base}[1]{\constr{B[} \; #1 \; \constr{]}}
\newcommand{\channel}[2]{\constr{C[} \; #1 \; \constr{;} \; #2 \; \constr{]}}
\newcommand{\comma}{\; \constr{,} \;}

% Functions
\newcommand{\subst}[3]{#1 \; \func{[} \; #3 \; \func{\mapsto} \;#2 \; \func{]}}
\newcommand{\op}[3]{#1 \; \func{\coloneqq} \; #2 \; \func{\cdot} \; #3}
\newcommand{\opsquared}[3]{#1 \, \func{\coloneqq} \, #2 \, \func{\cdot^2} \, #3}
\newcommand{\opctx}[3]{#1 \, \func{\coloneqq} \, #2 \, \func{\otimes} \, #3}
\newcommand{\zero}{\func{0\cdot}}
\newcommand{\one}{\func{1\cdot}}
\newcommand{\li}{\func{\ell_i}}
\newcommand{\lo}{\func{\ell_o}}
\newcommand{\lz}{\func{\ell_{\o}}}
\newcommand{\lio}{\func{\ell_{\#}}}

% Types
\newcommand{\Set}{\type{SET}}
\newcommand{\reduce}[1]{\; \type{\longrightarrow}_{#1} \;}
\newcommand{\types}[4]{#1 \; \type{;} \; #2 \; \type{\vdash} \; #3 \; \type{\triangleright} \; #4}
\newcommand{\contains}[6]{#1 \; \type{;} \; #2 \; \type{\ni}_{#3} \; #4 \; \type{;} \; #5 \; \type{\triangleright} \; #6}
\newcommand{\containsusage}[4]{#1 \; \type{\ni}_{#2} \; #3 \; \type{\triangleright} \; #4}
\newcommand{\Var}{\type{VAR}}
\newcommand{\Process}{\type{PROCESS}}
\newcommand{\Unused}{\type{UNUSED}}
\newcommand{\PreCtx}{\type{PRECTX}}
\newcommand{\Ctx}{\type{CTX}}
\newcommand{\Type}{\type{TYPE}\;}
\newcommand{\Idx}{\type{IDX}\;}
\newcommand{\Idxs}{\type{IDXS}}
\newcommand{\Usage}{\type{USAGE}}
\newcommand{\N}{\type{\mathbb{N}}}
\newcommand{\Channel}{\type{CHANNEL}}
\newcommand{\Rec}{\type{REC}}
\newcommand{\Coalgebra}{\type{COALGEBRA}}
\newcommand{\eq}[1]{\; \type{\simeq}_{#1} \;}
\newcommand{\eqeq}{\; \type{\cong} \;}

\usetheme[sectionpage=none,numbering=none]{metropolis}
\setbeamertemplate{title separator}{}
\date{}

\title{mechanising\\ the linear \picalc}
\author{\textbf{Uma Zalakain} \ and \  Ornela Dardha \\ University of Glasgow}

\begin{document}
  \maketitle

  \note{
    Hello everyone! I am Uma, and this talk is about my first experience formalising a type system (in this case a resource-aware \picalc{}) in Agda.
    This is joint work with my PhD supervisor Ornela Dardha.
    If you have any questions or comments please interrupt me at any time.
  }

  \begin{frame}{Overview}
    \begin{itemize}
      \setlength\itemsep{1em}
      \sitem the \picalc
      \begin{itemize}
        \sitem introduction
        \sitem syntax
        \sitem semantics
      \end{itemize}
      \sitem the linear \picalc
      \begin{itemize}
        \sitem introduction
        \sitem leftover type system
        \sitem type safety
      \end{itemize}
      \sitem a co-contextual reformulation
    \end{itemize}
  \end{frame}

  \begin{frame}{Notation}
    \begin{equation*}
      \begin{aligned}
        & \type{TYPES}          && \text{blue violet, uppercased, indices as subscripts} \\
        & \constr{constructors} && \text{burnt orange} \\
        & \func{functions}      && \text{olive green} \\
        & variables             && \text{black, in italics} \\
      \end{aligned}
    \end{equation*}
  \end{frame}

  \begin{frame}{The \picalc{}}
    \begin{itemize}
      \sitem a process algebra
      \sitem models concurrent and communicating processes
      \sitem processes communicate over channels
      \sitem channels are sent over channels (\emph{channel mobility})
      \sitem communication drives computation
    \end{itemize}
  \end{frame}

  \note{
    A little introduction to the \picalc{}, for those of you that are less familiar with it.
    The \picalc{} is a foundational model for communication, sort of like the \lambdacalc{} is for computation.
    The \picalc{} models processes communicating with one another over communication channels.
    It features channel mobility: channels can be sent over channels. 
    Process doing the sending and process doing the receiving can be syntactically far from each other.
    More on this later, when we introduce the syntax.

    In the linear \picalc{} channels must be used exactly once: they can't be discarded, they can't be duplicated.
    Why is this interesting? It provides us more fine-grained control over communication and allows us to state more interesting type safety properties.
    They are sufficient as a target for encoding the session-typed \picalc{}.

    This project comes from the idea of creating a common base framework that can be used to mechanise a bunch of type systems that are built on top of the \picalc{}, and so it's our first baby step.
  }

  \begin{frame}{Well-Scoped Syntax}
    \begin{mathpar}
      \inferrule
      {n : \N}
      {\nzero : \Var_{\suc n}}

      \inferrule
      {x : \Var_n}
      {\suc x : \Var_{\suc n}}
    \end{mathpar}

    \begin{equation*}
      \begin{aligned}
        \Process_n ::=& \; \PO_n               &&\text{(inaction)}    \\ 
        |& \; \new{} \Process_{\suc n}         &&\text{(restriction)} \\ 
        |& \; \comp{\Process_n}{\Process_n}    &&\text{(parallel)}     \\ 
        |& \; \recv{\Var_n}{}\Process_{\suc n} &&\text{(input)}       \\ 
        |& \; \send{\Var_n}{\Var_n}\Process_n  &&\text{(output)}       \\           
      \end{aligned}
    \end{equation*}

    \centering
    e.g.
    \begin{equation*}
      \small
      \begin{aligned}
        & (\new{}x ) && (\comp {\recvp{x}{y} && \sendp{y}{a} && \PO} {(\new{} y) && (\sendp{x}{y} && \recvp{y}{z} && \PO)})
        \\
        & \new{} && (\comp {\recv{0} && \send{0}{a} && \PO} {\new{} && (\send{1}{0} && \recv{0} && \PO)})
        \\
      \end{aligned}
    \end{equation*}
  \end{frame}

  \note[itemize]{
    \sitem \picalc{} syntax, well-scoped by construction through de Bruijn indices
    \sitem go over the constructors
  }

  \begin{frame}{Structural Congruence}
    \begin{mathpar}
      \inferrule
      { }
      {\constr{comp-assoc} : \comp{P}{(\comp{Q}{R})} \eqeq \comp{(\comp{P}{Q})}{R}}

      \inferrule
      { }
      {\constr{comp-comm} : \comp{P}{Q} \eqeq \comp{Q}{P}}
      
      \inferrule
      { }
      {\constr{comp-end} : \comp{P}{\PO_n} \eqeq P}
      
      \inferrule
      { }
      {\constr{scope-end} : \new \PO_{\suc n} \eqeq \PO_n}
      
      \inferrule
      {uQ : \Unused_{\nzero} \; Q}
      {\constr{scope-ext} : \new (\comp{P}{Q}) \eqeq \comp{(\new P)}{\func{lower}_{\nzero} \; \; Q \; uQ}}

      \inferrule
      { }
      {\constr{scope-comm} : \new \new P \eqeq \new \new \func{swap}_{\nzero} \; P}
    \end{mathpar}

    \centering
    $\eq{}$ is the congruent equivalence closure of $\eqeq$
  \end{frame}

  \note[itemize]{
    \sitem Parallel composition is associative, commutative and has $\PO$ as neutral element
    \sitem Garbage collection
    \sitem Scope extrusion (keep a proof $uQ$ alongside)
    \sitem Commutativity of scopes
  }

  \begin{frame}{Reduction Relation}
    \begin{mathpar}
      \inferrule
      { }
      {\constr{internal} : \Channel_n}

      \inferrule
      {i : \Var_n}
      {\constr{external} \; i : \Channel_n}

      \inferrule
      {i \; j : \Var_n \\ P : \Process_{\suc n} \\ Q : \Process_n}
      {\constr{comm} : \comp{\recv{i}P}{\send{i}{j}{Q}} \reduce{\constr{external} \; i} \comp{\func{lower}_{\nzero} \; (\subst{P}{\suc j}{\nzero}) \; uP'}{Q}}

      \inferrule
      {red : P \reduce{c} P'}
      {\constr{par} \; red : \comp{P}{Q} \reduce{c} \comp{P'}{Q}}

      \inferrule
      {red : P \reduce{c} Q}
      {\constr{res} \; red : \new P \reduce{\func{dec}\; c} \new Q}

      \inferrule
      {eq : P \eq{} P' \\ red : P' \reduce{c} Q}
      {\constr{struct} \; eq \; red : P \reduce{c} Q}
    \end{mathpar}
  \end{frame}

  \note[itemize]{
    \sitem Same scope before and after reduction
    \sitem Scope restriction is preserved (can be eliminated in strictly linear systems)
    \sitem Common channel $i$
    \sitem We keep track of $i$ at the type level
    \sitem Goes under parallel composition and scope restriction
    \sitem $\func{dec}$ saturates on $\constr{internal}$
  }

  \begin{frame}[t]
    \begin{columns}[T]
      \begin{column}{0.5\textwidth}
        \centering
        \vspace*{10em}
        \textcolor{gray}{untyped syntax \\ operational semantics}
        \vspace*{10em}
      \end{column}
      \vrule width 2pt
      \begin{column}{0.5\textwidth}
        \centering
        \vspace*{10em}
        \framebox{linear \picalc{}}
        \vspace*{10em}
      \end{column}
    \end{columns}
  \end{frame}
    
  \begin{frame}{The Linear \Picalc}
    \begin{itemize}
      \sitem channels must be used exactly once
      \sitem communication privacy
      \sitem no race conditions
      \sitem serves as a target encoding for session types
    \end{itemize}
  \end{frame}
    
  \begin{frame}{Typing with Leftovers}
    \begin{itemize}
      \sitem syntax directed
      \sitem independent type and usage contexts
      \sitem extra leftover usage context
      \sitem intrinsic context splits
      \sitem parametrised by a set of usage coalgebras
    \end{itemize}
  \end{frame}
  
  \note[itemize]{
  \sitem Only choice is in the types and usage annotations on scope restriction
  \sitem A syntax directed type system allows for inversion lemmas
  \sitem More on intrinsic/extrinsic later
  \sitem Linear, graded and shared types are all instances of an usage algebra
  }

  \begin{frame}{Usage Coalgebra}
    \begin{columns}
      \begin{column}{0.3\textwidth}
        \begin{itemize}
          \sitem partial
          \sitem neutral element $\zero$
          \sitem minimal element $\zero$
          \sitem associative
          \sitem commutative
          \sitem deterministic
          \sitem cancellative
          \sitem decidable
        \end{itemize}
      \end{column}
      \begin{column}{0.7\textwidth}
        \begin{equation*}
          \begin{aligned}
            &\zero                  &:{} &                 &        & C \\
            &\one                   &:{} &                 &        & C \\
            &\op{\_}{\_}{\_}        &:{} &                 &        & C \to C \to C \to \Set \\
          \end{aligned}
        \end{equation*}
        \centering
        \begin{tabular}{l | l | l}
          & carrier & operation \\
          \hline
          \textbf{linear} & \makecell[cl]{$\constr{0} \, : \, \type{Lin}$ \\ $\constr{1} \, : \, \type{Lin}$} & \makecell[cl]{$\op{\constr{0}}{\constr{0}}{\constr{0}}$ \\ $\op{\constr{1}}{\constr{1}}{\constr{0}}$ \\ $\op{\constr{1}}{\constr{0}}{\constr{1}}$} \\
          \hline
          \textbf{graded} & \makecell[cl]{$\nzero \, : \type{Gra}$ \\ $\suc \, : \type{Gra} \to \type{Gra}$} & \makecell[cl]{$\forall \, x \, y \, z$ \\ $\to x \, \type{\equiv} \, y \, \func{+} \, z$ \\ $\to \op{x}{y}{z}$} \\
          \hline
          \textbf{shared} & $\constr{\omega} \, : \, \type{Sha}$ & $\op{\constr{\omega}}{\constr{\omega}}{\constr{\omega}}$ \\
        \end{tabular}
      \end{column}
    \end{columns}
  \end{frame}

  \note[itemize]{
    \sitem Model as a ternary relation
    \sitem Decidable, deterministic, cancellative and has a minimal element
  }

  \begin{frame}{Indexed Usage Coalgebras}
    \begin{equation*}
      \begin{aligned}
        &\Idx               &: \; &\Set \\
        &\type{\exists IDX} &: \; &\Idx \\
        &\Usage             &: \; &\Idx \to \Set \\
        &\type{COALGEBRAS}    &: \; &(idx : \Idx) \to \Coalgebra_{\Usage_{idx}} \\
      \end{aligned}
    \end{equation*}

  \end{frame}

  \note[itemize]{
    \sitem Lump multiple algebras together
  }
  
  \begin{frame}{Capability Notation}
    \textbf{capability} input / output
    \hspace{1em}
    \textbf{multiplicity} $\zero$, $\one$, \ldots

    \begin{equation*}
      \begin{aligned}
        &C^{\func{2}} &&= C \type{\times} C \\
        &\lz          &&= \zero \comma \zero \\
        &\li          &&= \one \comma \zero \\
        &\lo          &&= \zero \comma \one \\
        &\lio         &&= \one \comma \one \\
        &\opsquared{(x_l \comma x_r)}{(y_l \comma y_r)}{(z_l \comma z_r)} &&= (\op{x_l}{y_l}{z_l}) \times (\op{x_r}{y_r}{z_r}) \\
      \end{aligned}
    \end{equation*}
  \end{frame}

  \note[itemize]{
    \sitem Notation borrowed from linear \picalc{}
    \sitem We use two carriers per channel: one for input, one for output
  }
  
  \begin{frame}{Types}
    \begin{mathpar}
      \inferrule
      { }
      {\unit : \Type}

      \inferrule
      {n : \N}
      {\base{n} : \Type}
      
      \inferrule
      {t : \Type \\ \stacked{idx : \Idx \\\\ x : \Usage_{idx}^{\func{2}}}}
      {\channel{t}{x} : \Type}
    \end{mathpar}

    \centering
    \vfill{}
    \textit{e.g.} $\channel{\channel{\unit}{\constr{\omega}}}{\li}$
  \end{frame}

  \begin{frame}{Typing Relation}
    \begin{adjustwidth}{-1.5em}{-1.5em}
    \begin{description}
    \item [$\PreCtx_n$] list of $\Type$s of length $n$
    \item [$\Idxs_n$] list of $\Idx$s of length $n$
    \item [$\Ctx_{idxs}$] list of $\Usage^{\func{2}}$s indexed over $idxs  \; : \Idxs_n$
    \end{description}
    
    \begin{mathpar}
    {\small
    \datatype{
      \gamma : \PreCtx_n \\
      \stacked{
        idxs : \Idxs_n \\\\
        \Gamma : \Ctx_{idxs}} \\
      P : \Process_n \\
      \Delta : \Ctx_{idxs}}
    {\types{\gamma}{\Gamma}{P}{\Delta} : \Set}}

    {\small
    \mprset {sep=.8em}
    \datatype{
      \gamma : \PreCtx_n \\
      \stacked{
        idxs : \Idxs_n \\\\
        \Gamma : \Ctx_{idxs}} \\
      i : \Var_n \\
      t : \Type \\
      \stacked{
        idx : \Idx \\\\
        y : \Usage_{idx}^{\func{2}}} \\
      \Delta : \Ctx_{idxs}}
    {\contains{\gamma}{\Gamma}{i}{t}{y}{\Delta} : \Set}}
    \end{mathpar}
    \end{adjustwidth}
  \end{frame}

  \note[itemize]{
  \sitem Typing judgment through typing relations
  \sitem One relation on processes, one relation on variable references (omitted for brevety)
  \sitem Relation on variable references uses the monoid to split multiplicities
  }
  
  \begin{frame}{Variable Typing Rules}
    \begin{mathpar}
      \inferrule
      {\opsquared{x}{y}{z}}
      {\nzero : \contains{\gamma \comma t}{\Gamma \comma x}{\nzero}{t}{y}{\Gamma \comma z}}
      
      \inferrule
      {\hspace{-0.1em} loc_i : \contains{\gamma \hspace{1.4em}}{\Gamma \hspace{1.7em}}{\hspace{1em} i}{t}{x}{\Delta}}
      {\suc \; loc_i : \contains{\gamma \comma t'}{\Gamma \comma x'}{\suc i}{t}{x}{\Delta \comma x'}}
    \end{mathpar}
  \end{frame}

  \note[itemize]{
    \sitem Constructor completely determined by $i$
    \sitem User alleviated from proof burden $\opsquared{x}{y}{z}$, which can be computed given $x$ and $y$
  }
  
  \begin{frame}{Process Typing Rules}
    \begin{mathpar}
      \inferrule
      { }
      {\PO : \types{\gamma}{\Gamma}{\PO}{\Gamma}}

      \inferrule
    {l \hspace{0.1em} : \types{\gamma}{\Gamma \hspace{0.3em}}{P}{\Delta} \\\\
      r : \types{\gamma}{\Delta}{Q}{\Xi}}
    {\comp{l}{r} : \types{\gamma}{\Gamma}{\comp{P}{Q}}{\Xi}}
    \end{mathpar}
  \end{frame}
  
  \note[itemize]{
  \sitem Instead of one authoritative context split, give decision power to P
  }

  \begin{frame}{Process Typing Rules}
    \begin{mathpar}
    \inferrule
    {t : \Type \\ x : \Usage_{idx}^{\func{2}} \\ y : \Usage_{idx'} \\\\
      cont : \types{\gamma \comma \channel{t}{x}}{\Gamma \comma (y \comma y) }{P}{\Delta \comma \lz}}
    {\new  \; t \; x \; y \; cont : \types{\gamma}{\Gamma}{\new P}{\Delta}}

    \inferrule
        {\stacked{
            chan_i : \contains{\gamma \hspace{1.1em}}{\Gamma \hspace{1.5em}}{i}{\channel{t}{x}}{\li}{\Xi} \\\\
            cont \hspace{0.4em} : \types{\gamma \comma t}{\Xi \comma x}{P \hspace{4.6em}}{\Theta \comma \lz}}}
        {\recv{chan_i}{cont} : \types{\gamma}{\Gamma}{\recv{i}{P}}{\Theta}}
  
    \inferrule
        {\ldots}
        {\send{chan_i}{loc_j}{cont} \; \ldots}
  
    \end{mathpar}
  \end{frame}

  \note[itemize]{
  \sitem Newly introduced channels are balanced
  \sitem The multiplicities of new variables must be exhausted
  }

  \begin{frame}[fragile]{Example Derivation}
    \begin{lstlisting}[mathescape]
      $\func{p}$ : $\Process_{\suc \constr{zero}}$
      $\func{p}$ = $\new{}(\comp{\recv{\nzero} (\recv{\nzero} \PO)}$
            ${\new{} (\send{\suc \; \nzero} {\nzero} \send{\nzero}{\suc \; \suc \; \nzero} \PO)})$

      $\func{\_}$ : $\types{\constr{[]} \comma \unit}{\constr{[]} \comma \func{\omega}}{\func{p}}{\func{\epsilon}}$
      $\func{\_}$ = $(\new \; \channel{\unit}{\func{\omega}} \; \li \; \one) \; (\comp{\recv{\nzero}{(\recv{\nzero}{\PO}})}$
            ${(\new \; \unit \; \func{\omega} \; \one) \; (\send{\suc \; \nzero}{\nzero}{\send{\nzero}{\suc \; \suc \; \nzero}{\PO}})})$
    \end{lstlisting}
  \end{frame}

  \begin{frame}{Before Leftover Typing}
    \begin{itemize}
    \sitem using functions to update usage contexts
      \begin{mathpar}
        \inferrule
        {\ldots \\ (\func{update}_i \; \ldots \; \Gamma) \vdash P \\ \ldots}
        {\Gamma \; \type{\vdash} \; \recv{i}{P}}
      \end{mathpar}
    \sitem extrinsic context splits
      \begin{mathpar}
        \inferrule
        {\opctx{\Gamma}{\Delta}{\Xi} \\\\
         \Delta \; \type{\vdash} P \\\\
         \Xi \; \type{\vdash} Q}
        {\Gamma \; \type{\vdash} \; \comp{P}{Q}}
      \end{mathpar}
    \end{itemize}
  \end{frame}

  \note[itemize]{
  \sitem The extrinsic context split is top-down, P already knows what it needs
  }

  \begin{frame}{Subject Congruence}
    \begin{itemize}
      \setlength\itemsep{1em}
      \sitem \textbf{framing} the only resources the well-typedness of a process depends on are the ones used by it.

      \sitem \textbf{weakening} inserting a new variable into the context preserves the well-typedness of a process as long as the usage annotation of the inserted variable is preserved as a leftover.

      \sitem \textbf{strengthening} removing an unused variable preserves the well-typedness of a process.

      \sitem \textbf{subject congruence} applying structural congruence rules to a well typed process preserves its well-typedness.
    \end{itemize}
  \end{frame}

  \note[itemize]{
  \sitem First three by induction on \textsc{Types} and \textsc{VarRef}
  \sitem First three state for the first time for the \picalc{}
  }

  \begin{frame}{Subject Reduction}
    \begin{itemize}
      \setlength\itemsep{2em}
      \sitem \textbf{renaming} \\
      substituting a variable $i$ for a variable $j$ in a well-typed process $P$ results in a well-typed process as long as the leftovers at index $j$ after $P$ are at least the consumption made by $P$ at index $i$.

      \sitem \textbf{subject reduction} \\
      let $\types{\gamma}{\Gamma}{P}{\Xi}$ and $P \reduce{c} Q$,
      \begin{itemize}
        \sitem if $c$ is $\constr{internal}$, then $\types{\gamma}{\Gamma}{Q}{\Xi}$.
        \sitem if $c$ is $\constr{external} \; i$ and $\containsusage{\Gamma}{i}{\lio}{\Delta}$, then $\types{\gamma}{\Delta}{Q}{\Xi}$.
      \end{itemize}
    \end{itemize}
  \end{frame}

  \begin{frame}[t]
    \begin{columns}[T]
      \begin{column}{0.5\textwidth}
        \centering
        \vspace*{10.5em}
        \textcolor{gray}{linear \picalc{}}
        \vspace*{10em}
      \end{column}
      \vrule width 2pt
      \begin{column}{0.5\textwidth}
        \centering
        \vspace*{10em}
        \minibox[frame]{co-contextual \\ reformulation}
        \vspace*{10em}
      \end{column}
    \end{columns}
  \end{frame}
  
  \begin{frame}{Co-Contextual Reformulation}
    \begin{itemize}
      \sitem goal: decidable typechecking
      \sitem how? infer the types of channels
      \sitem bidirectional typing does not apply
      \sitem traverse terms bottom up collecting equations on types
      \sitem solvable equations result in tautologies or substitutions
      \sitem prove sound and complete wrt the usual type system
      \sitem ongoing work (get in touch if you are interested!)
    \end{itemize}
  \end{frame}

  \begin{frame}
    \centering
    \Huge{thank you!} \\
    \Huge{\textasciicircum{}\_\textasciicircum{}} \\
    \Huge{questions?}
  \end{frame}
\end{document}